\section{Requisitos del sistema}

Para poder utilizar \textbf{FAQ Manager} ser� necesario disponer de un servidor de p�ginas Web (por ejemplo, Apache HTTP Server \cite{apache-homepage}) que pueda procesar c�digo PHP \cite{php-homepage}. Adem�s, ser� necesario un sistema gestor de bases de datos como MySQL \cite{mysql-homepage}. No obstante, para simplificar el proceso de instaci�n y configuraci�n de todo este software, se recomienda el uso de WAMP \cite{wamp-homepage}.

Una vez instalado y configurado el software necesario, deber� editar el fichero \texttt{config.php} que se incluye en la distribuci�n de \textit{FAQ Manager}. En este fichero, de estructura similiar a la de Figura \ref{fig:config}, deber� cumplimentar debidamente el dominio sobre el que se aloja el sistema gestor de bases de datos, el nombre de la base de datos, el nombre de un usuario de la base de datos con los permisos b�sicos para realizar operaciones \textit{CRUD} (Create, Read, Update, Delete) y su contrase�a.

\imagen{./imagenes/config.png}{0.7}{Fichero \texttt{config.php}}{fig:config}
