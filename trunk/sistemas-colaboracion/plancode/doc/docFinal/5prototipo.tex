\section{Prototipo de interfaz gr�fica de usuario}

Atendiendo a los requisitos que debe tener la herramienta \textit{PlanCoDE} comentados en la secci�n \ref{requisitos}, la interfaz gr�fica de usuario de la aplicaci�n ser� como la que muestra el prototipo de la Figura \ref{fig:prototipo}.

\imagen{./imagenes/prototipo}{0.3}{Proptotipo de la interfaz gr�fica de PlanCoDE}{fig:prototipo}

Los elementos que componen la interfaz son:

\begin{enumerate}
	\item \textbf{Barra de herramientas}: en este elemento se mostrar�n los diferentes men�s de la aplicaci�n.
	\item \textbf{Caja de herramientas}: aqu� se mostrar�n las herramientas necesarias para poder dibujar trazos y objetos sobre las im�genes de los mapas.
	\item \textbf{Lista de usuarios}: en este elemento se muestran los usuarios que est�n utilizando la aplicaci�n. 
	\item \textbf{Terminal}: en la primera pesta�a de este elemento se muestran los mensajes que se envi�n al chat, y en la segunda, se muestra todo el \textit{log} de acciones que realizan los usuarios.
	\item \textbf{�rea de trabajo}: aqu� se muestra la imagen del mapa seleccionado.
	\item \textbf{Barra de estado}: este elemento proporciona informaci�n al usuario acerca de la acci�n que est� realizando.
\end{enumerate}