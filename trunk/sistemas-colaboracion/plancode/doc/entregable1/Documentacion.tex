% Tipo de documento. En este caso es un art�culo, para folios A4, tama�o de la fuente 11pt y con p�gina separada para el t�tulo
\documentclass[a4paper,11pt,titlepage]{article}

% Carga de paquetes necesarios. OrdenesArticle es un paquete personalizado
\usepackage[spanish]{babel} 
\RequirePackage[T1]{fontenc}
\RequirePackage[ansinew]{inputenx} 
\usepackage[spanish,cap,cont,title,fancy]{OrdenesArticle}
\usepackage{lmodern}
\usepackage{array}
\usepackage{graphicx}
\usepackage{hyperref}
\usepackage{pifont}
\usepackage{listings}
\usepackage[usenames,dvipsnames]{color}
\usepackage{colortbl}
\usepackage{color}
\usepackage{ifthen}
\usepackage{longtable}
\hypersetup{bookmarksopen,bookmarksopenlevel=4,linktocpage,colorlinks,urlcolor=blue,citecolor=blue,
						linkcolor=blue,filecolor=blue,pdfnewwindow,
						pdftitle={PlanCoDE: Planificador Colaborativo para el Dise�o de Estrategias y Acciones de Emergencia},
						pdfauthor={Juan Andrada Romero, Jose Domingo L�pez L�pez},
						pdfsubject={Sistemas para la Colaboraci�n}}


% Macro para definir una lista personalizada 
\newenvironment{milista}%
{\begin{list}{\textbullet}%
{\settowidth{\labelwidth}{\textbullet} \setlength{\leftmargin}{\dimexpr\labelsep+\labelwidth+5pt}
\setlength{\itemsep}{\dimexpr 0.5ex plus 0.25ex minus 0.25ex}
\setlength{\parsep}{\itemsep}
\setlength{\partopsep}{\itemsep}
\addtolength{\topsep}{-7.5pt}
}}%
{\end{list}}

% Macro para insertar una imagen
%       Uso: \imagen{nombreFichero}{Factor escala}{Caption (leyenda)}{Label (identificador para referenciarla)}
% -------------------------------------------------------------------------------------------------------------
\def\imagen#1#2#3#4{
 \begin{figure}[h]
 \begin{center}
   \scalebox{#2}{\includegraphics{#1}}
 \caption {#3}
 \label{#4}
 \end{center}
 \end{figure}
}

% Macro para dar color a las filas de la tabla "`Operaciones"'
\def\colorFila{
	\rowcolor[rgb]{0.4,0.52,1.0}
}

% Definicion del "listings" para el lenguaje Java
\lstdefinelanguage{Java}
{
 morecomment = [l]{//}, 
 morecomment = [l]{///},
 morecomment = [s]{/*}{*/},
 morestring = [b]", 
 sensitive = true,
 morekeywords = {package, static, while, switch, break, line, void, String, Object, int, Integer, instanceof, else, if, for, private, return, new, public, class, import, int, boolean, true, false, extends, final, super, protected, abstract, this, do, float, double, null, try, catch, implements}
}

% Configuraci�n del c�digo incluido para el lenguaje Java
\lstset{
  language=Java,
  basicstyle=\footnotesize,
  backgroundcolor=\color{white},
  showspaces=false,
  showstringspaces=false,
  showtabs=false,
  frame=single,
  tabsize=2,
  captionpos=b,
  breaklines=true,
  breakatwhitespace=false,
  escapeinside={\%},
  keywordstyle = \color [rgb]{0,0,1},
  commentstyle = \color [rgb]{0.133,0.545,0.133},
  stringstyle = \color [rgb]{0.627,0.126,0.941}
}

\begin{document}

% En las p�ginas de portada e �ndices, no hay encabezado ni pie de p�gina
\pagestyle{empty} 

% Se incluye la portada
\thispagestyle{empty}
\begin{center}
  {\LARGE UNIVERSIDAD DE CASTILLA-LA MANCHA} \\
  \bigskip
  {\Large ESCUELA SUPERIOR DE INFORM�TICA} \\
  \vspace{28mm}
  \includegraphics[scale=0.45, keepaspectratio]{./imagenes/esi_bw.png} \\
  \vspace{30mm}
  {\LARGE \textbf{Sistemas para la Colaboraci�n}} \\
  \vspace{10mm}
  {\large \textsf{\textsc{- PlanCoDE -\\
  Planificador Colaborativo para el Dise�o de Estrategias y Acciones de Emergencia}}} \\
  \vspace{10mm}
  {\LARGE \textsf{\textsc{An�lisis de requisitos}}} \\
  \vspace{20mm}
  {\large Juan Andrada Romero} \\
  {\large Jose Domingo L�pez L�pez} \\
  \vspace{9mm}
  {\large \today}
\end{center}
\clearpage

% Texto del reverso de la portada
%%\mbox{}
%%\vspace{18cm}
%%\begin{small}
% Se ajusta la separaci�n entre p�rrafos
%%\parskip=10pt 

%%\copyright~ 2008/2009 Juan Andrada Romero. Universidad de Castilla La Mancha, Escuela Superior de Inform�tica de Ciudad Real.

%%Se permite la modificaci�n, copia y distribuci�n de este documento, seg�n la licencia de documentaci�n GNU (\url{http://www.gnu.org}).

%%Este documento fue compuesto con \LaTeX{}. Im�genes generadas con Power Point y Gimp.
%%\end{small}

%%\newpage

% En las p�ginas de �ndices y prefacio, se utiliza numeraci�n romana
\pagenumbering{Roman}

% Se crea el �ndice
\tableofcontents
% Se pasa p�gina y se a�ade el �ndice de figuras al �ndice principal
%\clearpage\phantomsection
%\addcontentsline{toc}{section}{\listfigurename}
% Se crea el �ndice de figuras
%\listoffigures
% Si se quiere crear un �ndice de tablas se pondr�a: \listoftables

\newpage

% Se ajusta la separaci�n entre p�rrafos
\parskip=10pt

% Se a�ade el prefacio al �ndice
%\clearpage\phantomsection
%\addcontentsline{toc}{section}{Prefacio}

% Comienza el contenido del documento. Se utilizan n�meros ar�bigos y el encabezado y pie de p�gina personalizado
\pagenumbering{arabic}
\pagestyle{fancy}

\section{Introducci�n}

El objetivo de este documento es realizar un estudio acerca de la herramienta colaborativa \textbf{Skype}. Dicho estudio se centrar� en clasificar la herramienta \textit{groupware} seg�n diferentes clasificaciones, y en detallar los diferentes requisitos funcionales de un sistema CSCW que cumple dicha herramienta. 

De este modo, se comenzar� en el apartado \ref{Skype} con una breve introducci�n a la herramienta, para continuar con la clasificaci�n (secci�n \ref{clasificacion}) y la descripci�n de sus requisitos funcionales (secci�n \ref{requisitos}). Para terminar, se comentar� en la secci�n \ref{manual} un peque�o manual del funcionamiento de Skype y, en la secci�n \ref{criticas}, se realizar�n algunas cr�ticas sobre aspectos que se podr�an mejorar en la herramienta.

\subsection{Introducci�n a la herramienta} \label{Skype}

Skype (\cite{skype}, \cite{wiki}) es un sistema que tiene la finalidad de conectar a los usuarios a trav�s de texto (mensajer�a instant�nea), voz o v�deo. Utiliza el protocolo VoIP para poder realizar llamadas y videoconferencias entre los distintos usuarios. Por este motivo, las llamadas son gratuitas cuando se realizan a trav�s de Internet, es decir, cuando la llamada se realiza a trav�s de Skype. Sin embargo, tambi�n es posible realizar una llamada desde Skype a un tel�fono convencional, aunque la llamada ya no es gratuita y se aplican diferentes tarifas (de menor coste que las tarifas de las distintas compa��as telef�nicas), seg�n el pa�s de destino.

As�, Skype permite (\cite{wiki-esp}, \cite{about}):

\begin{milista}
    \item Comunicaci�n por texto desde usuario Skype a usuario Skype (sin coste).
    \item Comunicaci�n por v�deo desde usuario Skype a usuario Skype (sin coste).
    \item Comunicaci�n por voz desde usuario Skype a usuario Skype (sin coste).
    \item Comunicaci�n por voz desde usuario Skype a tel�fono de red fija (contrato mensual y anual a muy bajo coste).
    \item Comunicaci�n por voz desde usuario Skype a tel�fono m�vil (contrato mensual y anual a muy bajo coste).
    \item Comunicaci�n de datos desde usuario Skype a fax de red fija o fax de PC (contrato mensual y anual a muy bajo coste).
		\item Comunicaci�n por voz desde tel�fono de red fija a n�mero telef�nico Skype (contrato mensual y anual a muy bajo coste).
    \item Comunicaci�n por voz desde tel�fono m�vil a n�mero telef�nico Skype (contrato mensual y anual a muy bajo coste).
		\item Comunicaci�n por fax desde un fax de red telef�nica a n�mero telef�nico Skype (contrato mensual y anual a muy bajo coste).
		\item Comunicaci�n por desv�o telef�nico y de texto desde un tel�fono de red fija o m�vil hacia un usuario de Skype (contrato mensual y anual a muy bajo coste).
\end{milista}

Skype sigue una arquitectura cliente-servidor, de modo que los usuarios tienen que registrarse y loguearse en el servidor para poder utilizar la aplicaci�n. Adem�s, con la evoluci�n en las tecnolog�as y la aparici�n de los dispositivos m�viles, Skype se puede utilizar actualmente desde diferentes dispositivos, gracias al desarrollo de \textit{Skype Mobile} (\cite{mobile}) para este tipo de dispositivos. 

Para terminar, hay que se�alar que en Skype se puede hablar del concepto \textbf{grupo de comunicaci�n}. Cuando Skype se utiliza entre usuarios que ejecutan dicha aplicaci�n, ya sea la aplicaci�n de escritorio o m�vil (es decir, no considerando el caso en que Skype se usa para realizar llamadas a la red telef�nica convencional), varios usuarios pueden crear un grupo y todo ese grupo se puede comunicar a trav�s del chat e incluso participar todos juntos en la misma llamada de voz. 
\section{�Qu� es PlanCoDE?} \label{requisitos}

PlanCoDE es el acr�nimo de \textit{Planificador Colaborativo para el Dise�o de Estrategias y Acciones de Emergencia}. Es, por tanto, una herramienta que permite la colaboraci�n de diferentes especialistas y cuerpos de seguridad y emergencia (polic�as, bomberos, etc.) para dise�ar y ejecutar un plan de actuaci�n y poder atender una situaci�n de emergencia. 

PlanCoDE ofrece las siguientes funcionalidades a sus usuarios:

\begin{milista}
	\item \textbf{Chat}: los diferentes miembros de grupos especialistas en situaciones de emergencia podr�n utilizar el chat que la aplicaci�n proporciona para comunicarse en tiempo real e intercambiar opiniones acerca del plan a realizar.
	\item \textbf{Visualizaci�n de mapas}: la aplicaci�n mostrar� una imagen del mapa de la zona donde se ha producido la emergencia. Dichos mapas se recuperar�n a partir de \textit{Google Maps} o ser�n im�genes ya almacenadas en disco. 
	\item \textbf{Edici�n de mapas}: los usuarios del sistema podr�n realizar trazos libres y colocar diferentes s�mbolos (barreras, puntos conflictivos, etc.) sobre la imagen del mapa que se ha cargado. De este modo, se pueden realizar rutas sobre el propio mapa y marcar con s�mbolos aquellos puntos de inter�s. 
	Del mismo modo, tambi�n se podr� eliminar los trazos y objetos introducidos por cada usuario de la aplicaci�n.	
\end{milista}

Adem�s, PlanCoDE debe tener como requisitos adicionales la \textbf{movilidad} y la \textbf{tolerancia a fallos} debido a las situaciones cr�ticas en las que dicha herramienta ser� utilizada. Esto implica disponer de una arquitectura lo m�s descentralizada posible. Por ello, �sta no ser� la t�pica arquitectura cliente-servidor, ya que si en alg�n momento el servidor no est� disponible por la raz�n que sea, los usuarios no podr�an iniciar el sistema para dise�ar las estrategias pertinentes.  Por tanto, el primer usuario que inicie una sesi�n se desempe�ar� la funci�n de servidor y cliente al mismo tiempo permitiendo que resto de usuarios puedan conectarse a �l. De modo que cuando se ejecuta la aplicaci�n, �sta debe permitir al usuario crear una nueva sesi�n o unirse a una sesi�n existente. En el segundo caso, ser� necesario introducir la direcci�n IP y el puerto al que debe conectarse la aplicaci�n. Para esto, el usuario podr� introducir manualmente dicha direcci�n y puerto, pero tambi�n podr� recuperar una lista de servidores (entendiendo servidor como uno de los clientes que iniciaron la sesi�n por primera vez) a trav�s de un servicio Web e iniciar sesi�n en el servidor deseado.

Con esta arquitectura, se satisfacen los requisitos cr�ticos de movilidad y tolerancia a fallos, pues cualquier cliente puede conectarse en cualquier momento y lugar y siempre va a existir alg�n servidor disponible, ya que �ste ser� uno de los propios clientes de la aplicaci�n. Adem�s, si el servicio Web falla al recuperar la lista de servidores, existir�a la posibilidad de introducir la informaci�n para conectarse con otros clientes de manera manual.


\subsection{Diagrama de casos de uso}

El diagrama de casos de uso de la aplicaci�n se muestra en la Figura \ref{fig:casosUso}. Dicho diagrama representa los requisitos de la aplicaci�n que se han comentado en el punto anterior, junto con otras funcionalidades que son necesarias para el correcto funcionamiento de la aplicaci�n colaborativa.

As�, un usuario debe iniciar una sesi�n para poder utilizar la aplicaci�n con el resto de usuarios y cierra la sesi�n cuando deja de utilizarla. Del mismo modo, la propia aplicaci�n debe actualizar su estado y enviar a todos los usuarios conectados esta informaci�n, es decir, los mensajes que se env�an al chat, la imagen del mapa que se desea mostrar y los dibujos que se realizan sobre �ste.

\imagen{./imagenes/casosUso}{0.7}{Diagrama de casos de uso de la aplicaci�n}{fig:casosUso}

\section{Clasificaci�n}

\subsection{Dimensiones espacio-temporal}
%johansen y grudin

\subsection{Dominio de aplicaci�n}
%ellis
\section{Descripci�n de los requisitos funcionales}
%decir el escenario que soporta (ver tema 1)
%dominio de aplicaic�n: colaboraci�n (no cooperaci�n) entre cuerpos especiales de seguridad y emergencia
\subsection{Participaci�n}

\subsection{Interacci�n}

\subsection{Coordinaci�n}
% control de concurrencia: chat (sin control), trazos (candados)
% control de concurrencia de acuerdo al manejo de los datos: duplicado
% selecci�n del m�todo de concurrencia: timeout para dibujar y bloqueo

\subsection{Distribuci�n y soporte}
% distribuci�n de los datos distribuida
% arquitectura de la implementaci�n: primero duplicada y futuro h�brida (ws)

\subsection{Notificaci�n de eventos}
%1 consciencia de uso compartido
%2 wysiwis relajado (s�lo area de trabajo)
%3 lista de usuarios
%  log de acciones
%  colores en los trazos asociados a los usuarios
%  avatar (en la lista de usuarios)
%  mostrar si un usuario est� escribiendo o dibujando (en la lista de usuarios)

\subsection{Visualizaci�n}
% interfaz de grupo wysiwis relajado
% tipo de ventana privada
% optimizaci�n del espacio -> toolbox en iconos
% Transmisi�n de los datos para mantener las vistas sincronizadas
% Actualizaci�n de la informaci�n
\section{Prototipo de interfaz gr�fica de usuario}

Atendiendo a los requisitos que debe tener la herramienta \textit{PlanCoDE} comentados en la secci�n \ref{requisitos}, la interfaz gr�fica de usuario de la aplicaci�n ser� como la que muestra el prototipo de la Figura \ref{fig:prototipo}.

\imagen{./imagenes/prototipo}{0.3}{Proptotipo de la interfaz gr�fica de PlanCoDE}{fig:prototipo}

Los elementos que componen la interfaz son:

\begin{enumerate}
	\item \textbf{Barra de herramientas}: en este elemento se mostrar�n los diferentes men�s de la aplicaci�n.
	\item \textbf{Caja de herramientas}: aqu� se mostrar�n las herramientas necesarias para poder dibujar trazos y objetos sobre las im�genes de los mapas.
	\item \textbf{Lista de usuarios}: en este elemento se muestran los usuarios que est�n utilizando la aplicaci�n. 
	\item \textbf{Terminal}: en la primera pesta�a de este elemento se muestran los mensajes que se envi�n al chat, y en la segunda, se muestra todo el \textit{log} de acciones que realizan los usuarios.
	\item \textbf{�rea de trabajo}: aqu� se muestra la imagen del mapa seleccionado.
	\item \textbf{Barra de estado}: este elemento proporciona informaci�n al usuario acerca de la acci�n que est� realizando.
\end{enumerate}

% Secci�n de ap�ndices
% Se a�ade el titulo de "Ap�ndices" al �ndice
%\addcontentsline{toc}{section}{Ap�ndices}

%\input{Apendices.tex}

% A�adimos la bibliografia al �ndice
\clearpage\phantomsection
\addcontentsline{toc}{section}{Referencias}

% Bibliograf�a. En este caso se usa BibTeX
%\bibliographystyle{plain}
%\pagestyle{plain} 
%\bibliography{Bibliografia}

\end{document}
