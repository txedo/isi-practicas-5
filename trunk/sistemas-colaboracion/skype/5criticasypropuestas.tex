\section{Cr�ticas y propuestas de posibles mejoras} \label{criticas}
Sin olvidar que Skype es una aplicaci�n muy completa y vers�til, se han detectado ciertos aspectos que podr�an mejorarse. B�sicaente, Se trata, entre otros, de aspectos de \textit{awareness} e interfaz gr�fica de usuario. A continuaci�n se enumeran y explican cada uno de ellos, y c�mo se podr�an rectificar:
\begin{milista}
	\item Cuando el usuario inicia sesi�n, inicialmente se muestra la lista de contactos con todos los contactos desconectados. Se debe esperar un tiempo relativamente largo para que la lista se actualice con el estado real de cada uno de los contactos. Este tiempo de espera puede provocar que el usuario piense que todos sus contactos est�n desconectados y finalice la sesi�n sin hacer uso de la aplicaci�n. Ser�a conveniente reducir la duraci�n de esta operaci�n para no provocar dicha sensaci�n en el usuario.
	\item La ventana destinada para mantener conversaciones mediante el chat, es utilizada para mostrar los mensajes de log relativos a las llamadas y videollamadas. Esto provoca que se cuele informaci�n irrelevante dentro de la conversaci�n que mantienen los participantes de la charla, haciendo que en ocasiones no se pueda leer al contacto de un modo confortable. Se propone que dichos mensajes de log sean mostrados en otra ventana distinta a la del chat, aunque se recomienda mantener en dicha ventana los mensajes de log correspondientes al env�o de ficheros.
	\item El mensaje de log generado al realizar una llamada es exactamente el mismo que el generado al realizar una videollamada. Esta caracter�stica puede provocar confusi�n en el usuario cuando �ste analice los mensajes de log ya que ambas operaciones son distintas (una llamada �nicamente implica audio y una videollamada implica audio y video) y deber�an generar mensajes de log personalizados.
	\item 
% El menu izquierdo es poco intuitivo porq al seleccionar un menu n se despliega hacia arriba como se espera, sino que aparece a la derecha.
	\item 
	% Cuando inicias una videoconferencia, existe una lista desplegable con diferentes opciones de video, pero sin texto y cerca del icono del volumen, lo que no tiene nada que ver y confunde al usuario.
	\item Una funcionalidad muy �til y que ha sido a�adida en las �ltimas versiones de la aplicaci�n, es la posibilidad de compartir la pantalla o una regi�n de �sta (ver secci�n \ref{sec:screenshare}. Esta caracter�stica se puede utilizar para retransmitir presentaciones, tutoriales, asistencia t�cnica, soporte, etc. No obstante, tiene una limitaci�n importante ya que es una retransmisi�n unidireccional, es decir, que el contacto \textit{A} puede enviar v�deo al contacto \textit{B} pero el contacto \textit{B} no puede enviarlo a \textit{A} de forma simult�nea. Ser�a interesante que se eliminase esta restricci�n permitiendo que la retransmisi�n pueda ser bidireccional.
	\item 
	% En las llamadas, el boton "`Comprobar configuracion"' lo que hace es ocultar iconos, que no tien mucho que ver con su texto. Adem�s, no varia el aspecto del bot�n cuando lo pulsas o no.
\end{milista}




