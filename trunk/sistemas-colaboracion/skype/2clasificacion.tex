\section{Clasificaci�n}

\subsection{Dimensiones espacio-temporal}
Atendiendo a la clasificaci�n en tiempo y espacio propuesta por Johansen, \textit{Skype} puede clasificarse de la siguiente forma:

%\begin{milista}
%	\item \textbf{Mismo lugar, diferente tiempo}: aunque el prop�sito principal de \textit{Skype} es la comunicaci�n por voz en tiempo real utilizando el protocolo VoIP (voz por IP), se puede clasificar en esta categor�a porque tambi�n permite que los usuarios utilicen el chat que dicha herramienta incorpora, recibiendo los mensajes a posteriori si el usuario se encuentra desconectado. Es por ello que se trata, en esete caso particular, de una \textbf{interacci�n as�ncrona}. ---- Realmente el chat es aqui o en \textbf{Diferente lugar, diferente tiempo}?? ------
%	\item \textbf{Diferente lugar, mismo tiempo}: \textit{Skype} se puede clasificar dentro de esta categor�a, ya que las llamadas de voz en tiempo real son s�ncronas, pero cada usuario puede encontrarse en diferentes lugares. Del mismo modo ocurre con el chat, cuando los usuarios participantes est�n conectados, ya que los mensajes enviados se env�an a todos ellos.
%\end{milista}


% CHAT: diferente lugar, mismo tiempo. Interacci�n distribuida sincrona. Usuarios conectados (mismo tiempo) y diferente lugar (cada uno est� en su casa)
% TELEFONO: diferente lugar, mismo tiempo. Interacci�n distribuida sincrona. Usuarios conectados (mismo tiempo) y diferente lugar (cada uno est� en su casa)
% Videoconferencia: diferente lugar, mismo tiempo. Interacci�n distribuida sincrona. Usuarios conectados (mismo tiempo) y diferente lugar (cada uno est� en su casa)

Por otra parte, atendiendo a la clasificaci�n espacio-temporal propuesta por Grudin, la herramienta se puede clasificar en las siguientes categor�as:
%
%\begin{milista}
%	\item \textbf{Mismo lugar, diferente tiempo (impredecible)}: ---- Como dice pizarra, ��entraria aqui el chat, como en el caso anterior?? -----
%	\item \textbf{Diferente lugar(impredecible), mismo tiempo}: --- Videoconferencia y llamadas
%\end{milista}

% Ambiguedad porq existe para dispositivos moviles, por lo que no hay una forma de clasificar si la aplicacion es predecible o impredecible. 
% Diferente lugar, mismo tiempo.

\subsection{Dominio de aplicaci�n}
%ellis

A: Es un sistema de mensaje porq se envian mensaje de texto (chat), sonido (llamadas) y video (videoconferencia)
D: Es un sistema de conferencias. Es del tipo D4: Conferencia de escritorio.






	
	