% Tipo de documento. En este caso es un art�culo, para folios A4, tama�o de la fuente 11pt y con p�gina separada para el t�tulo
\documentclass[a4paper,11pt,titlepage]{article}

% Carga de paquetes necesarios. OrdenesArticle es un paquete personalizado
\usepackage[spanish]{babel} 
\RequirePackage[T1]{fontenc}
\RequirePackage[ansinew]{inputenx} 
\usepackage[spanish,cap,cont,title,fancy]{OrdenesArticle}
\usepackage{array}
\usepackage{graphicx}
\usepackage{amsmath}
\usepackage{hyperref}
\usepackage{pifont}
\usepackage{listings}
\usepackage[usenames,dvipsnames]{color}
\usepackage{colortbl}
\usepackage{makeidx}
\hypersetup{bookmarksopen,bookmarksopenlevel=3,linktocpage,colorlinks,urlcolor=blue,citecolor=blue,
						linkcolor=blue,filecolor=blue,pdfnewwindow,
						pdftitle={Motor de indexaci�n}, 
						pdfsubject={Almacenamiento y Recuperaci�n de la Informaci�n}
						pdfauthor={Juan Andrada, Jose Domingo L�pez}}


% Macro para definir una lista personalizada 
\newenvironment{milista}%
{\begin{list}{\textbullet}%
{\settowidth{\labelwidth}{\textbullet} \setlength{\leftmargin}{\dimexpr\labelsep+\labelwidth+5pt}
\setlength{\itemsep}{\dimexpr 0.5ex plus 0.25ex minus 0.25ex}
\setlength{\parsep}{\itemsep}
\setlength{\partopsep}{\itemsep}
\addtolength{\topsep}{-7.5pt}
}}%
{\end{list}}


\begin{document}

% En las p�ginas de portada e �ndices, no hay encabezado ni pie de p�gina
\pagestyle{empty} 
% Se incluye la portada
\thispagestyle{empty}
\begin{center}
  {\LARGE UNIVERSIDAD DE CASTILLA-LA MANCHA} \\
  \bigskip
  {\Large ESCUELA SUPERIOR DE INFORM�TICA} \\
  \vspace{28mm}
  \includegraphics[scale=0.45, keepaspectratio]{./imagenes/esi_bw.png} \\
  \vspace{30mm}
  {\LARGE \textbf{Sistemas para la Colaboraci�n}} \\
  \vspace{10mm}
  {\large \textsf{\textsc{- PlanCoDE -\\
  Planificador Colaborativo para el Dise�o de Estrategias y Acciones de Emergencia}}} \\
  \vspace{10mm}
  {\LARGE \textsf{\textsc{An�lisis de requisitos}}} \\
  \vspace{20mm}
  {\large Juan Andrada Romero} \\
  {\large Jose Domingo L�pez L�pez} \\
  \vspace{9mm}
  {\large \today}
\end{center}

\pagenumbering{Roman}

\tableofcontents
\clearpage\phantomsection
\newpage

% Se ajusta la separaci\'on entre p\'arrafos
\parskip=10pt
\pagenumbering{arabic}
\pagestyle{fancy} 
% Aqui se incluyen los archivos .tex que forman el documento
\section{Introducci�n}
\subsection{Enunciado del problema}

\subsection{Dise�o elegido}
Para implementar la pr�ctica y, en concreto, este motor de indexaci�n, se utilizar� el lenguaje de programaci�n Python (\cite{Python}). 

Para implementar los elementos necesarios para la indexaci�n de documentos, como son el \textit{Posting$\_$File}, la tabla de documentos y el diccionario de t�rminos, se ha optado por utilizar una base de datos relacional, siguiendo el diagrama mostrado en la Figura \ref{BD}. Las tablas que se representan son:

\begin{milista}
	\item \textbf{dic}: implementa el diccionario de t�rminos. Tiene los campos \textit{term} y \textit{num$\_$docs}, que representan cada uno de los t�rminos encontrados, junto con el n�mero de documentos donde aparece cada t�rmino.
	\item \textbf{doc}: implementa la tabla de documentos. Tiene los campos \textit{id$\_$doc}, \textit{title} y \textit{path} para almacenar un identificador �nico de documento, el t�tulo del documento y la ruta del sistema donde se almacena una copia de dicho documento. 
	\item \textbf{posting$\_$file}: implementa el posting$\_$file. Contiene los campos \textit{term}, \textit{id$\_$doc} y \textit{frequency} para representar la frecuencia con la que aparece un t�rmino en un documento. Por tanto, el campo \textit{term} hace referencia a t�rminos del diccionario, y el campo \textit{id$\_$doc} hace referencia a los documentos de la tabla de documentos.
\end{milista}
\section{Decisiones de dise�o} \label{decisiones}

\subsection{Lenguaje de programaci�n y sistema operativo elegido}
Para implementar el sistema, se ha decidido utilizar el lenguaje de programaci�n Python (\cite{Python}), en un sistema operativo UNIX.

Se ha seleccionado este lenguaje de programaci�n ya que permite el uso de estructuras como diccionarios (tablas hash), listas, manejadores de archivos, llamadas al sistema, definici�n de patrones mediante expresiones regulares, etc. Dichos elementos han facilitado el desarrollo de este m�dulo del sistema, ya que Python los gestiona de una manera eficaz y permite su uso de uso de una manera sencilla.

Por otra parte, 

\subsection{Dise�o de la base de datos}

Para implementar los elementos necesarios para la indexaci�n de documentos, como son el \textit{Posting$\_$File}, la tabla de documentos y el diccionario de t�rminos, se ha optado por utilizar una base de datos relacional, siguiendo el diagrama mostrado en la Figura \ref{BD}. Las tablas que se representan son:

\begin{milista}
	\item \textbf{dic}: implementa el diccionario de t�rminos. Tiene los campos \textit{term} y \textit{num$\_$docs}, que representan cada uno de los t�rminos encontrados, junto con el n�mero de documentos donde aparece cada t�rmino.
	\item \textbf{doc}: implementa la tabla de documentos. Tiene los campos \textit{id$\_$doc}, \textit{title} y \textit{path} para almacenar un identificador �nico de documento, el t�tulo del documento y la ruta del sistema donde se almacena una copia de dicho documento. 
	\item \textbf{posting$\_$file}: implementa el posting$\_$file. Contiene los campos \textit{term}, \textit{id$\_$doc} y \textit{frequency} para representar la frecuencia con la que aparece un t�rmino en un documento. Por tanto, el campo \textit{term} hace referencia a t�rminos del diccionario, y el campo \textit{id$\_$doc} hace referencia a los documentos de la tabla de documentos.
\end{milista}

\begin{figure}[!tb]
	\centering
		\includegraphics[keepaspectratio]{BD}
	\caption{Diagrama de la base de datos}
	\label{fig:organigrama}
\end{figure} 


\clearpage
\section{Manual de usuario} \label{manual}

\subsection{Instalaci�n}
La aplicaci�n se ha desarrollado en Python y su interfaz gr�fica de
usuario en GTK. Adem�s, �sta requiere de los servicios de un servidor
MySQL. Dicho esto, el sistema en el cual se ejecutar� la aplicaci�n
debe tener instalado el siguiente software:
\begin{milista}
\item Python v2.5 o superior (\cite{Python})
\item GTK (\cite{GTK})
\item Psyco (\cite{psyco})
\item PyGTK (\cite{PyGTK})
\item MySQL (\cite{MySQL})
\item MySQLdb module for Python (\cite{python-mysqldb})
\end{milista}

Inicialmente se debe crear la base de datos que utilizar� el sistema
documental. En la distribuci�n del software se ofrece un fichero
``install'' bajo el directorio ra�z, que contiene las
sentencias necesarias para crear la base de datos y sus tablas
mediante el int�rprete de MySQL. Para acceder al int�rprete de
su servidor MySQL escriba la siguiente sentencia en un terminal:

\begin{center} \texttt{mysql -u root -p} \end{center}

Introduzca la contrase�a que configur� al instalar el servidor de MySQL y cree las tablas con el fichero comentado anteriormente. Llegados a este punto, ya tiene su sistema listo para ser utilizado, pero antes es necesario destacar que �ste consiste en dos aplicaciones que pueden ser utilizadas de forma conjunta o por separado: un motor de indexaci�n y un motor de b�squeda. Cada motor posee su propia interfaz gr�fica de usuario independiente y, adicionalmente, el motor de indexaci�n posee una interfaz basada en l�nea de comandos.

\subsection{Motor de indexaci�n}
\subsubsection{Interfaz gr�fica de usuario}
Para ejecutar la interfaz gr�fica sit�ese en el directorio
''presentaci�n'' y ejecute la sentencia
\texttt{./indexEngineGUI.py}. A continuaci�n se mostrar� la ventana principal, tal
y como aparece en la Figura \ref{fig:index-engine-gui}. Las opciones de dicha
ventana son:
\begin{milista}
	\item \textbf{Choose a file}: esta opci�n sirve para indexar
          un �nico fichero. Al pulsar el bot�n, se abrir� un cuadro de
          di�logo que permitir� al usuario seleccionar el fichero deseado. 
	\item \textbf{Choose a directory}: esta opci�n sirve para
          indexar todos los ficheros contenidos en el directorio
          dado. Al pulsar el bot�n correspondiente, se abrir� un nuevo
          di�logo que permitir� al usuario elegir un directorio. 
\end{milista}

\begin{figure}[h]
	\centering
		\includegraphics[keepaspectratio, scale=0.75]{./images/index-engine-gui}
	\caption{Interfaz gr�fica de usuario del motor de indexaci�n}
	\label{fig:index-engine-gui}
\end{figure} 

Una vez elegido un fichero o un directorio, basta con hacer clic en el
bot�n ''Start'' para que comienze la indexaci�n, mostrando una
barra de progreso para informar al usuario del avance de la indexaci�n de los
ficheros. 

\subsubsection{Interfaz de usuario por l�nea de comandos}
El m�dulo de indexaci�n adem�s cuenta con una interfaz basada en texto que puede
invocarse desde un terminal. Una vez m�s, esta interfaz est� contenida
en la capa de presentaci�n del sistema, por lo que hay que introducir la �rden (\texttt{./indexEngineCMD.py}), siendo necesario
proporcionarle dos par�metros, en funci�n de la operaci�n que queramos
realizar.
\begin{milista}
\item \texttt{[-f | --file] <ruta$\_$a$\_$un$\_$fichero>}. Para indexar un �nico fichero.
\item \texttt{[-d | --directory] <ruta$\_$a$\_$un$\_$directorio>}. Para indexar todos
  los ficheros alojados en un directorio
\end{milista}
Si es necesario, se puede consultar la ayuda con el par�metro ``-h'' o
``--help'' (ver Figura \ref{fig:menu-ayuda})

\begin{figure}[h]
	\centering
		\includegraphics[keepaspectratio, scale=0.55]{./images/menu-ayuda}
	\caption{Interfaz gr�fica de usuario}
	\label{fig:menu-ayuda}
\end{figure}

\subsection{Motor de b�squeda}
Debido a las funcionalidades que debe cubrir esta aplicaci�n, se proporciona una interfaz gr�fica de usuario (ver Figura \ref{fig:search-engine-gui}).

\begin{figure}[h]
	\centering
		\includegraphics[keepaspectratio, scale=0.55]{./images/search-engine-gui}
	\caption{Interfaz gr�fica de usuario del motor de b�squeda}
	\label{fig:search-engine-gui}
\end{figure}

En cuanto a las caracter�sticas de esta aplicaci�n cabe destacar las siguientes caracter�sticas:
\begin{milista}
\item Se puede invocar el motor de b�squeda a partir del men� ''Tools''.
\item En el men� ''Tools'' se pueden consultar los documentos indexados en el sistema (ver Figura \ref{fig:indexedDocuments}) y ver su contenido, haciendo doble clic en uno de ellos, o pulsando ENTER al seleccionarlo.
\begin{figure}[h]
	\centering
		\includegraphics[keepaspectratio, scale=0.55]{./images/indexedDocuments}
	\caption{Di�logo que muestra los documentos indexados en el sistema}
	\label{fig:indexedDocuments}
\end{figure}
\item A la hora de realizar una b�squeda podemos configurar los pesos que deseamos dar a cada t�rmino. Por defecto, cada t�rmino tendr� peso 1, que es el m�ximo que puede tener, de modo que si queremos quitarle importancia a un t�rmino debemos indicarle un peso entre 0 y 1 mediante su barra deslizante correspondiente. Otra caracter�stica importante del sistema es que nos permitir� configurar los pesos sobre los t�rmios que realmente formar�n parte de la pregunta, ignorando \textit{stopwords} y parseando el texto (ver Figura \ref{fig:custom-weights}).
\begin{figure}[h]
	\centering
		\includegraphics[keepaspectratio, scale=0.55]{./images/custom-weights}
	\caption{Configurando los pesos a los t�rminos parseados de una pregunta e ignorando stopwords}
	\label{fig:custom-weights}
\end{figure}
\item Una vez que hemos realizado una b�squeda, se muestran los documentos encontrados ordenados de mayor a mener relevancia con respecto a la pregunta introducida. Adem�s, si hacemos doble clic sobre un documento o lo seleccionamos y presionamos la tecla ENTER, emerger� una ventana mostr�ndonos su contenido (ver Figura \ref{fig:dialogoTexto}).
\begin{figure}[h]
	\centering
		\includegraphics[keepaspectratio, scale=0.55]{./images/dialogoTexto}
	\caption{Di�logo que muestra el texto del documento seleccionado}
	\label{fig:dialogoTexto}
\end{figure}
\item Tambi�n podemos buscar documentos similares al documento que tengamos seleccionado. Para ello haremos uso del bot�n ''Find similars''.
\item Por �ltimo, los resultados tambien pueden ser visualizados en el navegador configurado por defecto en el sistema (ver Figura \ref{fig:firefox}). Para ello haremos uso del boton ''Show XML''.
\begin{figure}[h]
	\centering
		\includegraphics[keepaspectratio, scale=0.55]{./images/firefox}
	\caption{Resultado de la b�squeda mostrado en el navegador Web}
	\label{fig:firefox}
\end{figure}

\end{milista}

\clearpage
\section{Estad�sticas de funcionamiento} \label{estadisticas}

Las pruebas se han realizado en un sistema Debian GNU/Linux 4.0
Etch sobre un ordenador Centrino@2.0GHz FSB800MHz 2.0GB RAM
DRR2@800MHz.

El conjunto de ficheros de muestra se compone de un total de 239.

Ya sea utilizando una memoria cach� de tama�o 1 (caso base, comportamiento
similar al de un sistema sin cach�) como de tama�o 1000, el tiempo
que se ha obtenido en la indexaci�n es similar: entorno a los 27
minutos. La diferencia radica en que en el primer caso el acceso a
disco y el uso del procesador se mantiene constante, mientras que en
el segundo caso ambos usos de recursos vienen dados por intevalos,
coincidiendo �stos con la sincronizaci�n de la cach� con la base de
datos.

No obstante, estos resultados nos hacen meditar en que el uso de la
memoria cach� no ha mejorado la eficiencia del sistema tanto como se
esperaba, por lo que en pr�ximas versiones se har�n las revisiones
pertinentes.

\clearpage

\clearpage\phantomsection
% A�adimos la bibliografia al \'indice
\addcontentsline{toc}{section}{Referencias}
% Bibliograf\'ia 
\bibliographystyle{plain}
\pagestyle{plain} 
\bibliography{Bibliografia}

\end{document}
