\section{Futuras mejoras}

\paragraph{Cache} Como se dice en la sección de estadísticas (ver
\ref{estadisticas}), la memoria caché no es tan óptima como se
esperaba. En próximas versiones se tratará de afinar la granuralidad
de esta funcionalidad para así minimizar, en la medida de lo posible,
el número de accesos a disco.

\paragraph{Extensibilidad} El sistema es completamente extensible
gracias a su arquitectura multicapa. Como prueba de ello se han
incorporado dos interfaces de usuario, una gráfica y otra basada en
texto, pero se quiere explotar más las posibilidades que nos brinda
esta arquitectura. Para ello se tratará de preparar la capa de
persistencia de modo que haya modos de almacenamiento alternativos a
la base de datos relacional que se utiliza actualmente.

\paragraph{Uso de disco} Aunque esta mejora ya está implícita en el
uso de memorias caché, se tratará de mejorar esta característica en
otro ámbito. Actualmente, cuando el sistema lee línea a línea el
fichero que está parseando para su indexación. El inconveniente de
este diseño está en que se están solicitando accesos a disco de forma
continua (uno por cada línea). Para reducir esta operación a un sólo
acceso a disco se leerá el fichero completo y se guardará en la
memoria RAM, de modo que el acceso a cada una de las líneas será más
eficiente. Esto es posible ya que los ficheros con los que trabaja el
sistema son de texto plano y no tienen un gran volumen.
