\section{Introducci�n}
\subsection{Enunciado del problema}

\subsection{Dise�o elegido}
Para implementar la pr�ctica y, en concreto, este motor de indexaci�n, se utilizar� el lenguaje de programaci�n Python (\cite{Python}). 

Para implementar los elementos necesarios para la indexaci�n de documentos, como son el \textit{Posting$\_$File}, la tabla de documentos y el diccionario de t�rminos, se ha optado por utilizar una base de datos relacional, siguiendo el diagrama mostrado en la Figura \ref{BD}. Las tablas que se representan son:

\begin{milista}
	\item \textbf{dic}: implementa el diccionario de t�rminos. Tiene los campos \textit{term} y \textit{num$\_$docs}, que representan cada uno de los t�rminos encontrados, junto con el n�mero de documentos donde aparece cada t�rmino.
	\item \textbf{doc}: implementa la tabla de documentos. Tiene los campos \textit{id$\_$doc}, \textit{title} y \textit{path} para almacenar un identificador �nico de documento, el t�tulo del documento y la ruta del sistema donde se almacena una copia de dicho documento. 
	\item \textbf{posting$\_$file}: implementa el posting$\_$file. Contiene los campos \textit{term}, \textit{id$\_$doc} y \textit{frequency} para representar la frecuencia con la que aparece un t�rmino en un documento. Por tanto, el campo \textit{term} hace referencia a t�rminos del diccionario, y el campo \textit{id$\_$doc} hace referencia a los documentos de la tabla de documentos.
\end{milista}