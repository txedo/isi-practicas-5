\section{Futuras mejoras}

\paragraph{Cache} Como se dice en la secci�n de estad�sticas (ver
\ref{estadisticas}), la memoria cach� no es tan �ptima como se
esperaba. En pr�ximas versiones se tratar� de afinar la granuralidad
de esta funcionalidad para as� minimizar, en la medida de lo posible,
el n�mero de accesos a disco.

\paragraph{Extensibilidad} El sistema es completamente extensible
gracias a su arquitectura multicapa. Como prueba de ello se han
incorporado dos interfaces de usuario, una gr�fica y otra basada en
texto, pero se quiere explotar m�s las posibilidades que nos brinda
esta arquitectura. Para ello se tratar� de preparar la capa de
persistencia de modo que haya modos de almacenamiento alternativos a
la base de datos relacional que se utiliza actualmente.

\paragraph{Uso de disco} Aunque esta mejora ya est� impl�cita en el
uso de memorias cach�, se tratar� de mejorar esta caracter�stica en
otro �mbito. Actualmente, cuando el sistema lee l�nea a l�nea el
fichero que est� parseando para su indexaci�n. El inconveniente de
este dise�o est� en que se est�n solicitando accesos a disco de forma
continua (uno por cada l�nea). Para reducir esta operaci�n a un s�lo
acceso a disco se leer� el fichero completo y se guardar� en la
memoria RAM, de modo que el acceso a cada una de las l�neas ser� m�s
eficiente. Esto es posible ya que los ficheros con los que trabaja el
sistema son de texto plano y no tienen un gran volumen.
