\section{Estad�sticas de funcionamiento} \label{estadisticas}

Las pruebas se han realizado en un sistema Ubuntu GNU/Linux 9.10 sobre un ordenador Intel Core 2 Duo@2.5GHz FSB800MHz 4.0GB RAM
DRR2@800MHz.

El conjunto de ficheros de muestra se compone de un total de 239 documentos.

\subsection{Motor de indexaci�n} 

En el primer motor de indexaci�n que se desarroll�, el tiempo que se obtuvo para indexar los 239 documentos fue de \textbf{27 minutos y 39 segundos}.

Con el cambio en el dise�o de la base de datos (ver secci�n \ref{basedatos}) y la optimizaci�n de las sentencias \textbf{insert} en la tabla \textit{dic} y en la tabla \textit{positng$\_$file}, el tiempo que se ha obtenido es de \textbf{4 minutos y 143141341 segundos}. Como se puede observar, la disminuci�n en el tiempo es bastante considerable.

Por otra parte, al igual que ocurr�a en la primera versi�n de este motor, la memoria cach� no disminuye el n�mero de acceso a disco. Esto es debido a que la cach� (configurada por defecto a 3000 l�neas) no llega a su m�xima capacidad, ya que debe vaciarse tras analizar cada documento, para mantener la integridad de clave ajena entre la tabla \textit{positng$\_$file} y la tabla \textit{dic}. Si los documentos tuviesen m�s t�rminos, la cach� llegar�a a ser efectiva. 


\subsection{Motor de b�squeda}

En la b�squeda, como se coment� en la secci�n \ref{basedatos}

\begin{table}[!ht]
	\renewcommand{\arraystretch}{1.35}
	\centering	
	\begin{tabular}{| p{3.8cm} | p{4.5cm} | p{5.5cm} | }
	\hline
	\textbf{Pregunta} & \textbf{Tiempo empleado} & \textbf{Documentos encontrados} \\
	\hline
	\end{tabular}
	\caption{Tabla resumen de los resultados de b�squedas}
	\label{tb:res}
\end{table}