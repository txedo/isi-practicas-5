\section{Estad�sticas de funcionamiento} \label{estadisticas}

Las pruebas se han realizado en un sistema Ubuntu GNU/Linux 9.10 sobre un ordenador Intel Core 2 Duo@2.5GHz FSB800MHz 4.0GB RAM
DRR2@800MHz.

El conjunto de ficheros de muestra se compone de un total de 239 documentos.

\subsection{Motor de indexaci�n} 

En el primer motor de indexaci�n que se desarroll�, el tiempo que se obtuvo para indexar los 239 documentos fue de \textbf{27 minutos y 39 segundos}.

Con el cambio en el dise�o de la base de datos (ver secci�n \ref{basedatos}) y la optimizaci�n de las sentencias \textbf{insert} en la tabla \textit{dic} y en la tabla \textit{positng$\_$file}, el tiempo que se ha obtenido es de \textbf{4 minutos y 27 segundos}. Como se puede observar, la disminuci�n en el tiempo es bastante considerable.

Por otra parte, al igual que ocurr�a en la primera versi�n de este motor, la memoria cach� no disminuye el n�mero de acceso a disco. Esto es debido a que la cach� (configurada por defecto a 3000 l�neas) no llega a su m�xima capacidad, ya que debe vaciarse tras analizar cada documento, para mantener la integridad de clave ajena entre la tabla \textit{positng$\_$file} y la tabla \textit{dic}. Si los documentos tuviesen m�s t�rminos, la cach� llegar�a a ser efectiva. 


\subsection{Motor de b�squeda}

En la b�squeda, como se coment� en la secci�n \ref{basedatos}, se ha optado por utilizar la �ltima sentencia \textbf{select}, apoy�ndose en la creaci�n de la vista temporal. Se ha elegido esta opci�n, porque, de media, tarda unos 5 segundos en buscar el t�rmino ''linux'' (que aparece en la mayor�a de los documentos), mientras que con la sentencia \textbf{select} que inclu�a una subconsulta, buscar el mismo t�rmino tardaba m�s de 40 segundos.

En la tabla \ref{tb:res} se muestra un resumen de b�squedas que se han realizado en el sistema (los tiempos son una media, ya que el tiempo de b�squeda puede variar seg�n la carga del procesador y la asignaci�n de �ste a los hilos). Como se puede observar, el tiempo var�a seg�n la cantidad de t�rminos introducidos y el n�mero de documentos en los que aparece ese t�rmino. 

\begin{table}[!ht]
	\renewcommand{\arraystretch}{1.35}
	\centering	
	\begin{tabular}{| p{4.3cm} | p{4.2cm} | p{5.5cm} | }
	\hline
	\textbf{Pregunta} & \textbf{Tiempo empleado} & \textbf{Documentos encontrados} \\
    \hline
    linux & 4.8 segundos & 237 documentos \\
    \hline
    LoopBack & 2.5 segundos & 33 documentos \\
    \hline
    0.0.0.25 & 0.22 segundos & 1 documento \\
    \hline
    127.0.0.1 & 2.1 segundos & 29 documentos \\
    \hline
    ETHERNET 127.0.0.1 NETWORK & 3.6 segundos & 177 documentos \\
    \hline
    cd-rom & 4.9 segundos & 167 documentos \\
    \hline
    \multicolumn{3}{|>{\columncolor[rgb]{0.8, 0.8, 0.8}}c|}{} \\
  	\hline
    Similar a ''Installation-HOWTO'' & 5.5 segundos & 239 documentos \\
    \hline
    Similar a ''CDROM-HOWTO'' & 5.7 segundos & 239 documentos \\
    \hline
    Similar a ''Assembly-HOWTO'' & 6.1 segundos & 239 documentos \\
    \hline
	\end{tabular}
	\caption{Tabla resumen de los resultados de b�squedas}
	\label{tb:res}
\end{table}


