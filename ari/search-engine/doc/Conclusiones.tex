\section{Conclusiones}\label{conclusiones}

Despu�s de la realizaci�n del sistema, se ha observado que es
pr�cticamente imposible llegar a encontrar una soluci�n perfecta al
problema, debido a que todo el desarrollo ha estado basado en la toma
de decisiones que mejoran el rendimiento del sistema por un lado y lo
empeoran por otro. En concreto, algunas acciones que mejoraban
considerablemente el m�dulo de indexaci�n provocaban deficiencias en
el m�dulo de b�squeda, y viceversa. Es por esto por lo que los
desarrolladores se han visto en la obligaci�n de llegar a un
compromiso en todo momento.

Uno de los principales problemas y cuyas decisiones han abarcado todo
el proceso de desarrollo fue la elaboraci�n del ''parser'', ya que el
dominio que maneja el sistema se basa en documentos t�cnicos escritos
en lenguaje natural y en diferentes idiomas. Un texto en lenguaje
natural se podr�a caracterizar por las pocas restricciones que impone
(p.e. despu�s de un punto va un espacio y letra may�scula) y los
frecuentes errores ortogr�ficos y gramaticales que pueden encontrarse
en ellos. Estos problemas condionan el dise�o de dicho ''parser''.

Adem�s, se ha comprobado un rendimiento excelente en el uso de MySQL
como sistema de almacenamiento. Es cierto que el m�dulo de
indexaci�n va experimentando cierto empeoramiento conforme el tama�o
de la base de datos va creciendo, pero a�n as� las estad�sticas
revelan tiempos muy aceptables para el volumen de documentos con el
que se trabaja. No obstante, el rendimiento del
sistema est� muy ligado al conocimiento de los desarrolladores del
lenguaje SQL y el dise�o de las sentencias de consulta y
escritura de la base de datos influye dr�sticamente en �ste.

Por �ltimo, como futuras mejoras a la aplicaci�n se recomienda
incorporar el uso de \emph{wordnets} y algoritmos de \emph{stemming},
con el objetivo de no perder la verdadera relevancia que tienen los
t�rminos a causa de sin�nimos, hip�nimos, hiper�nimos, palabras
derivadas, etc.