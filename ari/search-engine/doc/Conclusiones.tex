\section{Conclusiones}\label{conclusiones}

Después se la realización del sistema, se ha observado que es
prácticamente imposible llegar a encontrar una solución perfecta al
problema, debido a que todo el desarrollo ha estado basado en la toma
de decisiones que mejoran el rendimiento del sistema por un lado y lo
empeoran por otro. En concreto, algunas acciones que mejoraban
considerablemente el módulo de indexación provocaban deficiencias en
el módulo de búsqueda, y viceversa. Es por esto por lo que los
desarrolladores se han visto en la obligación de llegar a un
compromiso en todo momento.

Uno de los principales problemas y cuyas decisiones han abarcado todo
el proceso de desarrollo fue la elaboración del ``parser'', ya que el
dominio que maneja el sistema se basa en documentos técnicos escritos
en lenguaje natural y en diferentes idiomas. Un texto en lenguaje
natural se podría caracterizar por las pocas restricciones que impone
(p.e. después de un punto va un espacio y letra mayúscula) y los
frecuentes errores ortográficos y gramaticales que pueden encontrarse
en ellos. Estos problemas condionan el diseño de dicho ``parser''.

Además, se ha comprobado un rendimiento excelente en el uso de MySQL
como sistema de almacenamiento. Es cierto que el módulo de
indexación va experimentando cierto empeoramiento conforme el tamaño
de la base de datos va creciendo, pero aún así las estadísticas
revelan tiempos muy aceptables para el volumen de documentos con el
que se trabaja en este caso de estudio. No obstante, el rendimiento del
sistema está muy ligado al conocimiento de los desarrolladores del
lenguaje SQL y diseño de las sentencias de consulta y
escritura de la base de datos influye drásticamente en éste.

Por último, como futuras mejoras a la aplicación se recomienda
incorporar el uso de \emph{wordnets} y algoritmos de \emph{stemming},
con el objetivo de no perder la verdadera relevancia que tienen los
términos a causa de sinónimos, hipónimos, hiperónimos, palabras
derivadas, etc.