\section{Clasificaci�n}

\subsection{Dimensiones espacio-temporal}
Atendiendo a la clasificaci�n en tiempo y espacio propuesta por Johansen, \textit{Skype} puede clasificarse de la siguiente forma:


% CHAT: diferente lugar, mismo tiempo. Interacci�n distribuida sincrona. Usuarios conectados (mismo tiempo) y diferente lugar (cada uno est� en su casa)
% TELEFONO: diferente lugar, mismo tiempo. Interacci�n distribuida sincrona. Usuarios conectados (mismo tiempo) y diferente lugar (cada uno est� en su casa)
% Videoconferencia: diferente lugar, mismo tiempo. Interacci�n distribuida sincrona. Usuarios conectados (mismo tiempo) y diferente lugar (cada uno est� en su casa)

Por otra parte, atendiendo a la clasificaci�n espacio-temporal propuesta por Grudin, la herramienta se puede clasificar en las siguientes categor�as:
%

%	\item \textbf{Diferente lugar(impredecible), mismo tiempo}: --- Videoconferencia y llamadas
% Ambiguedad porq existe para dispositivos moviles, por lo que no hay una forma de clasificar si la aplicacion es predecible o impredecible. 
% Diferente lugar, mismo tiempo.

\subsection{Dominio de aplicaci�n}
%ellis

%A: Es un sistema de mensaje porq se envian mensaje de texto (chat), sonido (llamadas) y video (videoconferencia)
%D: Es un sistema de conferencias. Es del tipo D4: Conferencia de escritorio.



\subsection{Modelo 3C}

% Introdcucion
% Comentar la piramide (difumninar los que no cumplimos)






	
	