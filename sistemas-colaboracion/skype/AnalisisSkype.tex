% Tipo de documento. En este caso es un art�culo, para folios A4, tama�o de la fuente 11pt y con p�gina separada para el t�tulo
\documentclass[a4paper,11pt,titlepage]{article}

% Carga de paquetes necesarios. OrdenesArticle es un paquete personalizado
\usepackage[spanish]{babel} 
\RequirePackage[T1]{fontenc}
\RequirePackage[ansinew]{inputenx} 
\usepackage[spanish,cap,cont,title,fancy]{OrdenesArticle}
\usepackage{lmodern}
\usepackage{graphicx}
\usepackage{hyperref}
\usepackage[usenames,dvipsnames]{color}
\usepackage{color}
\usepackage{ifthen}
\usepackage{longtable}
\usepackage{enumerate}
\hypersetup{bookmarksopen,bookmarksopenlevel=4,linktocpage,colorlinks,urlcolor=blue,citecolor=blue,
						linkcolor=blue,filecolor=blue,pdfnewwindow,
						pdftitle={An�lisis de un sistema colaborativo: Skype},
						pdfauthor={Juan Andrada Romero, Jose Domingo L�pez L�pez},
						pdfsubject={Sistemas para la Colaboraci�n}}


% Macro para definir una lista personalizada 
\newenvironment{milista}%
{\begin{list}{\textbullet}%
{\settowidth{\labelwidth}{\textbullet} \setlength{\leftmargin}{\dimexpr\labelsep+\labelwidth+5pt}
\setlength{\itemsep}{\dimexpr 0.5ex plus 0.25ex minus 0.25ex}
\setlength{\parsep}{\itemsep}
\setlength{\partopsep}{\itemsep}
\addtolength{\topsep}{-7.5pt}
}}%
{\end{list}}

% Macro para insertar una imagen
%       Uso: \imagen{nombreFichero}{Factor escala}{Caption (leyenda)}{Label (identificador para referenciarla)}
% -------------------------------------------------------------------------------------------------------------
\def\imagen#1#2#3#4{
 \begin{figure}[h]
 \begin{center}
   \scalebox{#2}{\includegraphics{#1}}
 \caption {#3}
 \label{#4}
 \end{center}
 \end{figure}
}

\begin{document}

% En las p�ginas de portada e �ndices, no hay encabezado ni pie de p�gina
\pagestyle{empty} 

% Se incluye la portada
\thispagestyle{empty}
\begin{center}
  {\LARGE UNIVERSIDAD DE CASTILLA-LA MANCHA} \\
  \bigskip
  {\Large ESCUELA SUPERIOR DE INFORM�TICA} \\
  \vspace{28mm}
  \includegraphics[scale=0.45, keepaspectratio]{./imagenes/esi_bw.png} \\
  \vspace{30mm}
  {\LARGE \textbf{Sistemas para la Colaboraci�n}} \\
  \vspace{10mm}
  {\large \textsf{\textsc{- PlanCoDE -\\
  Planificador Colaborativo para el Dise�o de Estrategias y Acciones de Emergencia}}} \\
  \vspace{10mm}
  {\LARGE \textsf{\textsc{An�lisis de requisitos}}} \\
  \vspace{20mm}
  {\large Juan Andrada Romero} \\
  {\large Jose Domingo L�pez L�pez} \\
  \vspace{9mm}
  {\large \today}
\end{center}
\clearpage

% Texto del reverso de la portada
%%\mbox{}
%%\vspace{18cm}
%%\begin{small}
% Se ajusta la separaci�n entre p�rrafos
%%\parskip=10pt 

%%\copyright~ 2008/2009 Juan Andrada Romero. Universidad de Castilla La Mancha, Escuela Superior de Inform�tica de Ciudad Real.

%%Se permite la modificaci�n, copia y distribuci�n de este documento, seg�n la licencia de documentaci�n GNU (\url{http://www.gnu.org}).

%%Este documento fue compuesto con \LaTeX{}. Im�genes generadas con Power Point y Gimp.
%%\end{small}

%%\newpage

% En las p�ginas de �ndices y prefacio, se utiliza numeraci�n romana
\pagenumbering{Roman}

% Se crea el �ndice
\tableofcontents
% Se pasa p�gina y se a�ade el �ndice de figuras al �ndice principal
%\clearpage\phantomsection
%\addcontentsline{toc}{section}{\listfigurename}
% Se crea el �ndice de figuras
%\listoffigures
% Si se quiere crear un �ndice de tablas se pondr�a: \listoftables

\newpage

% Se ajusta la separaci�n entre p�rrafos
\parskip=10pt

% Se a�ade el prefacio al �ndice
%\clearpage\phantomsection
%\addcontentsline{toc}{section}{Prefacio}

% Comienza el contenido del documento. Se utilizan n�meros ar�bigos y el encabezado y pie de p�gina personalizado
\pagenumbering{arabic}
\pagestyle{fancy}

\section{Introducci�n}

El objetivo de este documento es realizar un estudio acerca de la herramienta colaborativa \textbf{Skype}. Dicho estudio se centrar� en clasificar la herramienta \textit{groupware} seg�n diferentes clasificaciones, y en detallar los diferentes requisitos funcionales de un sistema CSCW que cumple dicha herramienta. 

De este modo, se comenzar� en el apartado \ref{Skype} con una breve introducci�n a la herramienta, para continuar con la clasificaci�n (secci�n \ref{clasificacion}) y la descripci�n de sus requisitos funcionales (secci�n \ref{requisitos}). Para terminar, se comentar� en la secci�n \ref{manual} un peque�o manual del funcionamiento de Skype y, en la secci�n \ref{criticas}, se realizar�n algunas cr�ticas sobre aspectos que se podr�an mejorar en la herramienta.

\subsection{Introducci�n a la herramienta} \label{Skype}

Skype (\cite{skype}, \cite{wiki}) es un sistema que tiene la finalidad de conectar a los usuarios a trav�s de texto (mensajer�a instant�nea), voz o v�deo. Utiliza el protocolo VoIP para poder realizar llamadas y videoconferencias entre los distintos usuarios. Por este motivo, las llamadas son gratuitas cuando se realizan a trav�s de Internet, es decir, cuando la llamada se realiza a trav�s de Skype. Sin embargo, tambi�n es posible realizar una llamada desde Skype a un tel�fono convencional, aunque la llamada ya no es gratuita y se aplican diferentes tarifas (de menor coste que las tarifas de las distintas compa��as telef�nicas), seg�n el pa�s de destino.

As�, Skype permite (\cite{wiki-esp}, \cite{about}):

\begin{milista}
    \item Comunicaci�n por texto desde usuario Skype a usuario Skype (sin coste).
    \item Comunicaci�n por v�deo desde usuario Skype a usuario Skype (sin coste).
    \item Comunicaci�n por voz desde usuario Skype a usuario Skype (sin coste).
    \item Comunicaci�n por voz desde usuario Skype a tel�fono de red fija (contrato mensual y anual a muy bajo coste).
    \item Comunicaci�n por voz desde usuario Skype a tel�fono m�vil (contrato mensual y anual a muy bajo coste).
    \item Comunicaci�n de datos desde usuario Skype a fax de red fija o fax de PC (contrato mensual y anual a muy bajo coste).
		\item Comunicaci�n por voz desde tel�fono de red fija a n�mero telef�nico Skype (contrato mensual y anual a muy bajo coste).
    \item Comunicaci�n por voz desde tel�fono m�vil a n�mero telef�nico Skype (contrato mensual y anual a muy bajo coste).
		\item Comunicaci�n por fax desde un fax de red telef�nica a n�mero telef�nico Skype (contrato mensual y anual a muy bajo coste).
		\item Comunicaci�n por desv�o telef�nico y de texto desde un tel�fono de red fija o m�vil hacia un usuario de Skype (contrato mensual y anual a muy bajo coste).
\end{milista}

Skype sigue una arquitectura cliente-servidor, de modo que los usuarios tienen que registrarse y loguearse en el servidor para poder utilizar la aplicaci�n. Adem�s, con la evoluci�n en las tecnolog�as y la aparici�n de los dispositivos m�viles, Skype se puede utilizar actualmente desde diferentes dispositivos, gracias al desarrollo de \textit{Skype Mobile} (\cite{mobile}) para este tipo de dispositivos. 

Para terminar, hay que se�alar que en Skype se puede hablar del concepto \textbf{grupo de comunicaci�n}. Cuando Skype se utiliza entre usuarios que ejecutan dicha aplicaci�n, ya sea la aplicaci�n de escritorio o m�vil (es decir, no considerando el caso en que Skype se usa para realizar llamadas a la red telef�nica convencional), varios usuarios pueden crear un grupo y todo ese grupo se puede comunicar a trav�s del chat e incluso participar todos juntos en la misma llamada de voz. 
\clearpage
\section{Clasificaci�n}

\subsection{Dimensiones espacio-temporal}
Atendiendo a la clasificaci�n en tiempo y espacio propuesta por Johansen, \textit{Skype} puede clasificarse de la siguiente forma:

\begin{milista}
	\item \textbf{Mismo lugar, diferente tiempo}: aunque el prop�sito principal de \textit{Skype} es la comunicaci�n por voz en tiempo real utilizando el protocolo VoIP (voz por IP), se puede clasificar en esta categor�a porque tambi�n permite que los usuarios utilicen el chat que dicha herramienta incorpora, recibiendo los mensajes a posteriori si el usuario se encuentra desconectado. Es por ello que se trata, en esete caso particular, de una \textbf{interacci�n as�ncrona}. ---- Realmente el chat es aqui o en \textbf{Diferente lugar, diferente tiempo}?? ------
	\item \textbf{Diferente lugar, mismo tiempo}: \textit{Skype} se puede clasificar dentro de esta categor�a, ya que las llamadas de voz en tiempo real son s�ncronas, pero cada usuario puede encontrarse en diferentes lugares. Del mismo modo ocurre con el chat, cuando los usuarios participantes est�n conectados, ya que los mensajes enviados se env�an a todos ellos.
\end{milista}

Por otra parte, atendiendo a la clasificaci�n espacio-temporal propuesta por Grudin, la herramienta se puede clasificar en las siguientes categor�as:

\begin{milista}
	\item \textbf{Mismo lugar, diferente tiempo (impredecible)}: ---- Como dice pizarra, ��entraria aqui el chat, como en el caso anterior?? -----
	\item \textbf{Diferente lugar(impredecible), mismo tiempo}: --- Videoconferencia y llamadas
\end{milista}

\subsection{Dominio de aplicaci�n}
%ellis






	
	---- �Las compras se suponen interaccion?? -------
	---- ��Gestionar contactos se supone interaccion?? -----
	---- Mas funcionalidades .... --------
	
\clearpage
\section{Descripci�n de los requisitos funcionales}

\subsection{Participaci�n}

% Sesion que se refiere a la sesion del servidor

1.1. Predefinido porq hay q estar registrado e inicar sesion en el servidor
1.2. Permitir entradas posteriores en el contexto del servidor
Introducir usuarios al inicio de la sesi�n, en el contexto de crear grupos de chats, llamadas y videoconferencias. Tambien es permitir entradas posteriores.
1.3. Permitir salir antes.
1.4. Aceptada, porq el usuario tiene que aceptar la invitaci�n que envia uno de los usuarios del grupo.
1.5. Segun el contexto de uso de la aplicaci�n. Para un uso dom�stico, normalmente es de 2 participantes y, ocasionalmente, peque�o, pero para un uso profesional y empresarial, suele ser peque�o grande.
1.7. Conjunta. Ejemplo de chat colaborativo
1.8. Textual
Voz y video
1.10. Persistente el log de chats y de acciones. Tb los contactos
No persistente audio y video
1.11 Depende de la tarea (chat, llamada...)
1.12 Id�ntico en el chat, log y llamada
Diferente en la videoconferencia (pantallita de cara)

\subsection{Interacci�n}

2.1. Sincrono
2.2. Distribuida
2.3. Ambos, dependiendo del contexto de uso de la aplicaci�n (empresarial, domestico)
2.4. Ambos, dependiendo del contexto de uso de la aplicaci�n (empresarial, domestico)
2.5. Democr�tico, porq todos los usuarios pueden realizar las mismas acciones (ejemplo IRC:administradores, etc)
2.6. Ambos, dependiendo del contexto de uso de la aplicaci�n (empresarial, domestico)
2.7. Estructurada ?????
2.8. Identificada
2.9. Simultanea

\subsection{Coordinaci�n}

% No puede haber mucha cosas porq no es una aplicacion de coordinacion

\subsection{Distribuci�n y soporte}

4.1. Centralizada en el servidor (logs, perfiles, contactos, compras, credito, etc.)
4.2. Hibrida, porq las peticiones y el login pasn por el servidor pero al establecer ya la comunicaci�n, el procesamiento se realiza en los clientes (por ejemplo, el video)


\subsection{Notificaci�n de eventos}

5.1. Consciente (de la interaccion)
5.3. Lista de usuarios con iconos de estado

\subsection{Visualizaci�n}

6.1. WYSIWIS relajada (chat)
WYSIWIS relajada (llamadas, porq la duracion es la misma pero la configuracion de la llamada no)
No aplicable para videoconferencias
6.2. Privada (porq cada cliente ejecuta su aplicacion)
6.3. Ninguna
6.4. ????
6.5. Si, para el chat
6.6. Actualizar el estado de los usuarios, avatar, log de eventos

\clearpage
\section{Explicaci�n del funcionamiento} \label{manual}

Aunque este documento se ha centrado en las funcionalidades principales (ver secci�n \ref{sec:introduccion}) para las cuales fue desarrollada la aplicaci�n (comunicaci�n basada en chat, voz y videollamada), Skype goza de otras muchas caracter�sticas para optimizar la experiencia de usuario y que ser�n tratadas en esta secci�n.

A continuaci�n se enumerar�n y explicar�n las distintas funcionalidades adicionales que hacen de Skype una aplicaci�n excepcional. Se seguir� un orden l�gico, es decir, desde funcionalidades m�s b�sicas (para usuarios principiantes) a funcionalidades m�s espec�ficas (para usuarios expertos).


\subsection{Personalizaci�n del perfil de usuario}
Al instalar la aplicaci�n se ofrece al usuario la posibilidad de crear una cuenta de usuario. Este es un proceso obligatorio y gratuito, ya que s�lo los usuarios registrado pueden hacer uso del servicio. El proceso de registro es el t�pico: elegir un nombre de usuario, una contrase�a, y rellenar algunos datos personales como el nombre, la direcci�n y la fecha de nacimiento.

Una vez que el usuario ha creado la cuenta y ha iniciado sesi�n en el sistema, acceder� a la ventana principal de la aplicaci�n. En este lugar es posible modificar el perfil de usuario para introducir m�s datos personales y conectar la cuenta de Skype con la cuenta de MySpace (ver Figura \ref{fig:configurarPerfil}). Adem�s, se puede configurar el avatar (una foto que representa al usuario virtualmente) y un mensaje de estado que ser� visualizado por todos los contactos.

\imagen{./imagenes/configurarPerfil.png}{0.4}{Ventana de configuraci�n del perfil de usuario}{fig:configurarPerfil}


\subsection{Configuraci�n del idioma}
Actualmente, Skype est� disponible en 29 idiomas distintos (para ver la lista completa acuda al men� \textit{Herramientas/Cambiar idioma}). Gracias a esta caracter�stica la aplicaci�n puede llegar a una gran variedad de pa�ses y culturas. Esto es posible porque la configuraci�n del idioma de Skype se basa en un fichero de extensi�n \textit{.lang} en el que se escriben los nombres de las variables y sus valores. Por esto, Skype ofrece la posibilidad de crear nuevos archivos de idioma o editar los ya existentes, maximizando as� la \textit{internacionalizaci�n}.


\subsection{Gesti�n de contactos} \label{sec:contactos}
Una caracter�stica b�sica en todos los sistemas de comunicaciones es la gesti�n de contactos. Skype, como el resto de aplicaciones del estilo, permite al usuario la posibilidad de agregar nuevos contactos y organizarlos en grupos.

El proceso de agregaci�n de contactos puede hacerse de 3 formas:
\begin{milista}
	\item \textbf{Manual:} el usuario puede agregar un contacto a partir de su nombre Skype, su nombre real completo o su direcci�n de correo electr�nico.
	\item \textbf{Autom�tico:} el usuario indica a la aplicaci�n de qu� fuente desea agregar contactos y el sistema se encargar� de agregar autom�ticamente todos los contactos de dicha fuente existentes en el sistema Skype. La fuente puede ser Gmail, Facebook, AOL, Microsoft Outlook Express, etc. Esta funcionalidad recibe el nombre de \textit{Importar contactos}.
	\item \textbf{Asistido:} el usuario introduce en el buscador interno de Skype una serie de caracter�sticas. El buscador devolver� un conjunto de usuarios que cumplen con dichas caracter�sticas, permitiendo su agregado como contactos del usuario.
\end{milista}

Por �ltimo, Skype tambi�n ofrece la posibilidad de guardar y restaurar copias de seguridad de contactos.


\subsection{Compartir pantalla}\label{sec:screenshare}
Para maximizar las posibilidades que ofrece la videollamada, Skype implementa otra funcionalidad que consiste en compartir la pantalla del ordenador. Hay dos modos posibles:
\begin{milista}
	\item Compartir pantalla completa.
	\item Compartir selecci�n de pantalla.
\end{milista}

Ambos modos persiguen el mismo fin: permitir al usuario compartir una vista de la pantalla de su equipo o una selecci�n de �sta. Esta funci�n puede ser muy �til cuando se intenta de mostrar el modo de realizar alguna operaci�n en el ordenador (p.e. c�mo resolver un problema), o cuando se desea transmitir una presentaci�n mediante una videollamada.


\subsection{Compartir archivos}
Al igual que cualquier programa de mensajer�a, Skype ofrece la posibilidad de compartir ficheros entre los distintos contactos. En este aspecto no se ha innovado nada.


\subsection{Informaci�n de la calidad de las llamadas}



\subsection{Compra de cr�dito}
Como se indica en la secci�n \ref{sec:introduccion}, existen una serie de servicios \textbf{no gratuitos} que se pueden realizar en Skype (p.e. llamar desde la cuenta Skype a un tel�fono m�vil). La duda que puede tener el usuario a la hora de consumir este tipo de servicios es c�mo se realiza el pago de los mismos. Para ello, la aplicaci�n facilita una funcionalidad denominada \textit{Comprar}, accesible desde la parte inferior del men� izquierdo. Se muestran tres opciones de compra (ver Figura \ref{fig:comprar}:
\begin{milista}
	\item \textbf{Comprar cr�dito de Skype:} se abrir� una ventana dentro de la aplicaci�n en la que el usuario deber� introducir sus datos personales (nombre, apellidos, direcci�n de facturaci�n, etc.) y seleccionar el m�todo de pago: PayPal o tarjeta. A continuaci�n, el saldo seleccionado (m�nimo 10.00\euro) ser� sumado al cr�dito de la cuenta del usuario. Tambi�n se ofrece la opci�n de recargar autom�ticamente la cuenta de Skype cuando el saldo se sit�a por debajo de los 2.00\euro.
	\item \textbf{Comprar un plan:} los planes son un tipo de tarifa plana para llamar a tel�fonos fijos. En la Figura \ref{fig:planes} se muestran los distintos planes que hay ofertados para residentes europeos. Cabe destacar que esta funci�n no se realiza dentro de la propia aplicaci�n, si no que se ejecuta en el navegador Web que el equipo tenga configurado como predeterminado. El pago se realiza de la misma forma que en el caso anterior: PayPal o tarjeta.
	\item \textbf{Comprar accesorios:} como su propio nombre indica, en este lugar se facilita la compra de accesorios: desde auriculares y tel�fonos hasta c�maras Web. Una vez m�s, esta funcionalidad se ejecutar� en el navegador Web que haya configurado de forma predeterminada en el equipo y el pago se har� mediante PayPal o tarjeta.
\end{milista}

\imagen{./imagenes/comprar.png}{0.4}{Funcionalidad \textit{Comprar} de Skype}{fig:comprar}

\imagen{./imagenes/plan-europa.png}{0.4}{Planes de Skype ofertados en Europa}{fig:planes}


\subsection{Gesti�n del directorio}
Esta secci�n ofrece dos funcionalidades:
\begin{milista}
	\item La primera es una segunda v�a de acceso para agregar o importar nuevos contactos (ver secci�n \ref{sec:contactos})
	\item La segunda sirve para buscar recomendaciones de empresas. Para ello, se introduce el nombre de la empresa o una descripci�n de �sta (p.e. restaurante y pizzeria italiano) y la ciudad en la que se desea realizar la b�squeda. El sistema nos devolver� una lista de valoraciones que han hecho los usuarios sobre las empresas que se ajusten a las condiciones de b�squeda. Tambi�n se permite a�adir nuevas recomendaciones o valoraciones.
\end{milista}


\subsection{Extras}
Por �ltimo, y una de las funcionalidades m�s interesantes que ofrece Skype, es la posibilidad de instalar \textit{extras} en la aplicaci�n, a modo de \textit{plug-ins}, para ampliar y maximizar la experiencia de usuario. Los \textit{extras} est�n clasificados por categor�as:
\begin{milista}
	\item \textbf{Colaboraci�n:} contiene utilidades colaborativas que permiten tareas como las de crear \textit{meeting rooms} virtuales, pizarras multiusuario, compartir aplicaciones y archivos, calendarios colaborativos, etc.
	\item \textbf{Expresi�n:} se centra en aspectos sociales tales como compartir canciones con los contactos, a�adir \textit{smilies}, configurar im�genes animadas como avatar, etc.
	\item \textbf{Productividad:} entre otros, destacan extras que permiten grabar las conversaciones y las videollamadas, enviar faxes, controlar la aplicaci�n mediante la voz, etc.
	\item \textbf{Empresa:} destacan extras que hacen posible la transferencia de llamadas, la creaci�n de centros de llamadas, agendas de contactos multiusuario, etc.
	\item \textbf{Utilidades:} esta categor�a recoge una amplia variedad de extras que no son recogidos en el resto de categor�as. En ella, podemos encontrar utilidades para tirar dados online, recibir las recomendaciones de last.fm, enviar mensajes de texto a trav�s del tel�fono m�vil, etc.
	\item \textbf{Acceso remoto:} actualmente s�lo existe un extra en esta categor�a. Como se ha comentado en la secci�n \ref{clasificacion} Skye est� disponible tanto como aplicaci�n de escritorio como para dispositivos m�viles, pero no todos los dispositivos m�viles soportan la aplicaci�n. Para paliar este problema, existe un extra llamado \textit{HypeCall} que hace que sea posible realizar llamadas Skype desde cualquier tel�fono m�vil.
	\item \textbf{Comunidad:} esta categor�a est� orientada a aspectos sociales para establecer relaciones entre usuarios que no se conocen entre ellos.
	\item \textbf{Juegos:} como su propio nombre indica, en esta categor�a se ofrecen m�s de veinte juegos colaborativos para jugar en modo online con los contactos.
\end{milista}

\clearpage
\section{Cr�ticas y propuestas de posibles mejoras}

% Secci�n de ap�ndices
% Se a�ade el titulo de "Ap�ndices" al �ndice
%\addcontentsline{toc}{section}{Ap�ndices}

%\input{Apendices.tex}

% A�adimos la bibliografia al �ndice
\clearpage\phantomsection
\addcontentsline{toc}{section}{Referencias}

% Bibliograf�a. En este caso se usa BibTeX
\bibliographystyle{plain}
\pagestyle{plain} 
\bibliography{Bibliografia}

\end{document}
