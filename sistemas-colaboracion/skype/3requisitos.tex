\section{Descripci�n de los requisitos funcionales}

\subsection{Participaci�n}

% Sesion que se refiere a la sesion del servidor

1.1. Predefinido porq hay q estar registrado e inicar sesion en el servidor
1.2. Permitir entradas posteriores en el contexto del servidor
Introducir usuarios al inicio de la sesi�n, en el contexto de crear grupos de chats, llamadas y videoconferencias. Tambien es permitir entradas posteriores.
1.3. Permitir salir antes.
1.4. Aceptada, porq el usuario tiene que aceptar la invitaci�n que envia uno de los usuarios del grupo.
1.5. Segun el contexto de uso de la aplicaci�n. Para un uso dom�stico, normalmente es de 2 participantes y, ocasionalmente, peque�o, pero para un uso profesional y empresarial, suele ser peque�o grande.
1.7. Conjunta. Ejemplo de chat colaborativo
1.8. Textual
Voz y video
1.10. Persistente el log de chats y de acciones. Tb los contactos
No persistente audio y video
1.11 Depende de la tarea (chat, llamada...)
1.12 Id�ntico en el chat, log y llamada
Diferente en la videoconferencia (pantallita de cara)

\subsection{Interacci�n}

2.1. Sincrono
2.2. Distribuida
2.3. Ambos, dependiendo del contexto de uso de la aplicaci�n (empresarial, domestico)
2.4. Ambos, dependiendo del contexto de uso de la aplicaci�n (empresarial, domestico)
2.5. Democr�tico, porq todos los usuarios pueden realizar las mismas acciones (ejemplo IRC:administradores, etc)
2.6. Ambos, dependiendo del contexto de uso de la aplicaci�n (empresarial, domestico)
2.7. Estructurada ?????
2.8. Identificada
2.9. Simultanea

\subsection{Coordinaci�n}

% No puede haber mucha cosas porq no es una aplicacion de coordinacion

\subsection{Distribuci�n y soporte}

4.1. Centralizada en el servidor (logs, perfiles, contactos, compras, credito, etc.)
4.2. Hibrida, porq las peticiones y el login pasn por el servidor pero al establecer ya la comunicaci�n, el procesamiento se realiza en los clientes (por ejemplo, el video)


\subsection{Notificaci�n de eventos}

5.1. Consciente (de la interaccion)
5.3. Lista de usuarios con iconos de estado

\subsection{Visualizaci�n}

6.1. WYSIWIS relajada (chat)
WYSIWIS relajada (llamadas, porq la duracion es la misma pero la configuracion de la llamada no)
No aplicable para videoconferencias
6.2. Privada (porq cada cliente ejecuta su aplicacion)
6.3. Ninguna
6.4. ????
6.5. Si, para el chat
6.6. Actualizar el estado de los usuarios, avatar, log de eventos
