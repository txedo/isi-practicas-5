\section{Clasificaci�n} \label{clasificacion}

En los siguientes apartados, se realizar� una clasificaci�n de la herramienta PlanCode, seg�n los diferentes autores que aparecen en \cite{groupware}.

\subsection{Dimensiones espacio-temporal}

Atendiendo a la clasificaci�n en tiempo y espacio propuesta por \textbf{Johansen}, PlanCode puede clasificarse en la categor�a \textit{diferente lugar, mismo tiempo}, es decir, es una herramienta de interacci�n s�ncrona distribuida (ver Figura \ref{fig:johansen}). Esto es as� porque los diferentes usuarios deben estar conectados en la misma sesi�n y al mismo tiempo para poder recibir los mensajes del chat, la imagen del mapa cargada por alguno de ellos y los trazos que se van realizando.

\imagen{./imagenes/johansen.png}{0.3}{Taxonom�a en tiempo y espacio (Johansen; 1991)}{fig:johansen}

Por otra parte, atendiendo a la clasificaci�n espacio-temporal propuesta por \textbf{Grudin}, la herramienta se puede clasificar tambi�n en la categor�a \textit{diferente lugar, mismo tiempo}, por las mismas razones que en el caso anterior. 


\subsection{Dominio de aplicaci�n}

Siguiendo la clasificaci�n del nivel de aplicaciones de \textbf{Ellis}, PlanCode cumple con las caracter�sticas de los siguientes sistemas:
\begin{enumerate}[A.]
	\item \textbf{Sistema de mensajes}: PlanCode se puede considerar un sistema de mensajes porque permite el intercambio de mensajes de diferente naturaleza entre los usuarios, como es el intercambio de mensajes de texto a trav�s del chat, im�genes cuando se carga un mapa y el intercambio de trazos libres.
\end{enumerate}

\subsection{Modelo 3C}

Si atendemos al objetivo principal de PlanCode (el dise�o de rutas sobre un mapa en situaciones de emergencias), dicha herramienta se puede clasificar como una herramienta de colaboraci�n, pues, como ya se ha comentado, los trazos pueden ser dibujados por cualquier usuario, apoyando al resto de trazos que ya est�n dibujados. Del mismo modo, cualquier usuario podr�a modificar la imagen del mapa que se est� mostrando si lo cree conveniente. \\
\indent Pero, como PlanCode cuenta con un chat, se puede considerar tambi�n una herramienta de comunicaci�n y coordinaci�n.


