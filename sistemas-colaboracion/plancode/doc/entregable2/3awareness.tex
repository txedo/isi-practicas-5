\section{Soporte de \textit{awareness}}

En esta secci�n se va comentar el soporte \textit{awareness} del espacio de trabajo de la aplicaci�n, es decir, como se soporta la interacci�n entre los diferentes usuarios de la aplicaci�n. 
As�, se van detallar las t�cnicas empleadas para intentar responder a las preguntas que aparecen en \cite{tema5}. Cabe destacar que que no es necesario responder a todas ellas, sobrecargando incluso el sistema si se responden.

\begin{milista}
	\item \textbf{Identidad}: PlanCoDE mantendr� una lista de usuarios conectados, indicando su rol dentro del sistema. As�, se indicar� si el usuario corresponde a un cuerpo policial, a un cuerpo de bomberos, etc. De este modo, el resto de usuarios ser�n conscientes de qu� tipos de usuarios est�n haciendo uso de la herramienta. 
	Adem�s, cada usuario, y todas las acciones que �ste realice, se van a identificar por un color.
	\item \textbf{Acciones}: para indicar lo que cada usuario est� haciendo, se va a utilizar el concepto de \textit{telepuntero}, como ya se ha comentado en la secci�n \ref{sec:telepunteros}. 
	\indent Por otra parte, cuando un usuario comience a escribir en el chat, el resto de usuarios tambi�n ser�n informados de esta situaci�n, ya que se mostrar� un mensaje indicando que ese usuario est� escribiendo texto.
	\item \textbf{Intenciones}: cuando un usuario quiera utilizar el �rea de dibujo, debe solicitar el \textit{telepuntero}. De este modo, se informar� al resto de usuarios acerca de qui�n lo ha solicitado, pudiendo saber qui�n va a manipular el �rea de trabajo a continuaci�n.
	\item \textbf{Cambios}: relacionado con todo lo anterior, un usuario puede conocer los cambios que otro usuario est� haciendo sobre el �rea de dibujo, pues dichos cambios se ir�n mostrando al utilizar el \textit{telepuntero}. \\
	\indent Adem�s, todos los cambios realizados en el �rea de trabajo se almacenar�n tambi�n en forma de \textit{log} textual, por lo que cualquier usuario puede consultar las acciones que ha llevado a cabo otro usuario.
	\item \textbf{Objetos}: el objeto que se utiliza por parte de los usuarios es el \textit{telepuntero}, ya que para poder dibujar en el �rea de trabajo, previamente deben solicitar su turno para poder hacerlo.
\end{milista}

Para terminar, cabe destacar que para dotar a PlanCoDE de mayor funcionalidad, se ha incluido un canal adicional de texto, que es el chat, para que los usuarios puedan comunicarse en tiempo real. Es un canal adicional porque el principal objetivo de esta herramienta es el dise�o de un plan de emergencia realizando modificaciones sobre un �rea de dibujo, en la cu�l se carga una imagen o un mapa.