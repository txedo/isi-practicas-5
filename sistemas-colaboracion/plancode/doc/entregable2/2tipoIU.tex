\section{Interfaz WYSIWIS relajado}
%Explicar por qu� es una interfaz de tipo WYSIWIS y por qu� es de tipo relajado (ver traspas tema 5)
% SEPARACION DE ESPACIOS DE TRABAJO
% se relaja la barra de menus y las ventanas que se abren a partir de �l, el toolbox
% el wusiwis se aplica sobre el chat, log, ventana de dibujo y lista de usuarios

% VISUALIZACION DE CURSORES
% cada usuario tiene su cursor privado y no puede ver el de los dem�s, puesto que en un futuro se quiere portar la aplicaci�n a dispositivos m�viles y la pantalla no es apta para tal fin.

% GESTION DE LA DISTRIBUCI�N EN LA PANTALLA
% si

% GESTION DE LA INFORMACION VISUALIZADA
% aqu� creo que habr�a que meter que en el log se va mostrando en modo texto las acciones de dibujo que se realizan sobre el area de trabajo compartida (area de dibujo)

% ACOPLAMIENTO DE INTERFACES DE USUARIO
% creo que ser�a nivel medio

% TELEPUNTEROS
% podemos hacer que para dibujar haya que solicitar un telepuntero, si no, nada


