\section{Arquitectura de red de JXTA}

\subsection{Organizaci�n de la red}

La red JXTA es una red \textit{Ad-Hoc}, multi-salto y adaptativa, compuesta de pares conectados. Las conexiones en la red pueden ser temporales y, como resultado, el enrutamiento de mensajes entre pares es no determinista, ya que los pares pueden unirse o dejar la red en cualquier momento, lo que resulta siempre en un cambio de informaci�n de enrutamiento.

Normalmente se utilizan cuatro tipos de pares en una red JXTA, como se observa en la Figura \ref{fig:redJXTA}:

\begin{milista}
	\item \textbf{\textit{Minimal-Edge Peer}}: puede enviar y recibir mensajes, pero no guardar anuncios en cach� o enrutar mensajes para otros pares. 
	\item \textbf{\textit{Full-featured Peer}}: puede enviar y recibir mensajes y t�picamente guardara en cach� los anuncios. Este tipo de par simple contesta a peticiones de descubrimiento con informaci�n encontrada en su cach� de anuncios, pero no reenv�a ninguna petici�n de descubrimiento. 
	\item \textbf{\textit{Rendezvous peer}}: es un par de infraestructura que ayuda a otros pares con la propagaci�n de mensajes, descubrimiento de anuncios y rutas y mantiene un mapa topol�gico de otros pares de infraestructura, que se utilizan para controlar la propagaci�n y el mantenimiento de la tabla Hash distribuida. Cada grupo de pares mantiene su propio conjunto de pares \textit{rendezvous} y puede tener tantos como sea necesario. 
	\item \textbf{\textit{Relay peer}}: es un par de infraestructura que ayuda a otros pares no direccionables con la retransmisi�n del mensaje. Un par puede solicitar una secci�n del mensaje a este tipo de par para facilitar la retransmisi�n del mensaje cada vez que sea necesario.
\end{milista}

\imagen{./imagenes/arquitecturaRed}{0.4}{Arquitectura de red de JXTA}{fig:redJXTA}


\subsection{�ndice Distribuido de Recursos Compartidos (\textit{Shared Resource Distributed Index - SRDI})}

JXTA implementa un SRDI para conseguir un mecanismo eficiente de propagaci�n de peticiones en la red JXTA. Los pares de tipo \textit{rendezvous} mantienen un �ndice de anuncios publicados por los pares de tipo \textit{edge}, de tal modo que cu�ndo uno de estos pares publica un anuncio, el par \textit{rendezvous} lo indexa en su SRDI, utilizando como clave el ID del anuncio.

Cada par de tipo \textit{rendezvous} conoce a otros pares de este tipo que se encuentren en su grupo, de tal modo que cuando se indexa un nuevo recurso, el �ndice utilizado se env�a al resto de pares \textit{rendezvous}, para que tambi�n lo almacenen en su SRDI. 

A continuaci�n, se va a comentar un ejemplo de uso del SRDI cuando un par solicita un recurso, que es el mostrado en la Figura \ref{fig:ejSRDI}:

\imagen{./imagenes/ejemploSRDI}{0.2}{Propagaci�n de una petici�n en una red JXTA usando SRDI}{fig:ejSRDI}

\begin{milista}
	\item El par A de tipo \textit{edge} solicita un recurso a su par R1 de tipo \textit{rendezvous} y al resto de la red, usando difusi�n (\textit{multicast}).
	\item Si los pares de la red de A tienen el recurso en su cach�, se lo env�an.
	\item Del mismo modo, el par R1 usa el �ndice del anuncio de petici�n para buscarlo en su SRDI. Si lo encuentra, se lo env�a al par A. Si no lo encuentra, propaga la petici�n a los pares \textit{rendezvous} que se encuentran en su grupo, para que �stos realicen la b�squeda o repitan este proceso si no tienen ese recurso disponible.
\end{milista}


\subsection{Firewalls y NAT}

Un par detr�s de un cortafuegos (\textit{firewall}) puede enviar un mensaje directamente a un par que est� fuera del cortafuegos, pero no es posible establecer una comunicaci�n directa en el otro sentido. Lo mismo ocurre para los pares que se encuentren tras un dispositivo de NAT (\textit{Network Address Translation}).

Para poder realizar la conexi�n con pares que est�n tras cortafuegos o usan NAT, se deben cumplir las siguientes condiciones:

\begin{milista}
	\item Al menos un par del grupo que se encuentra en el cortafuego debe poder comunicarse con un par tras el cortafuegos.
	\item Estos pares deben soportar un protocolo com�n de transporte, como TCP o HTTP.
	\item El cortafuegos debe permitir tr�fico HTTP o TCP. 
\end{milista}


	
	