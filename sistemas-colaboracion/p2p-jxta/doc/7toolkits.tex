\section{Toolkits e implementaciones alternativas}
Aunque la implementaci�n de JXTA en Java es una buena herramienta para desarrollar aplicaciones P2P, no es la �nica alternativa de la que disponen los desarrolladores. Existen numerosas alternativas tanto de JXTA para otros lenguajes, como otra plataformas diferentes a JXTA.

\subsection{IBM BabbleNet}
\textit{AlphaWorks} es un grupo de IBM que desarrolla y distribuye nuevas tecnolog�as, principalmente para desarrolladores. BabbleNet \cite{babblenet} es una de estas tecnolog�as y consiste en un programa P2P descentralizado que permite a sus usuarios construir un chat en tiempo real sin la necesidad de conectarse a un servidor. El sistema est� desarrollado sobre un framekwork P2P escrito en Java que soporta comunicaciones entre nodos.

El c�digo fuente puede examinarse una vez instalado el software. La documentaci�n es m�nima y los comentarios que hay en el c�digo describen el sistema. Se trata de un software en fase experimental y con licencia \textit{open-source}.

\subsection{Intel}
Intel tambi�n se encuentra dentro del mercado del P2P, y lo hace con su \textit{Peer-to-Peer Acelerator Kit} \cite{p2p-acelerator-kit}. Se trata de un \textit{middleware} que se utiliza con Microsoft .NET.

\subsection{Microsoft .NET}
Microsoft ha desarrollado una p�gina Web en la que se muestra c�mo se pueden desarrollar sistemas P2P con la plataforma .NET \cite{net-p2p}. Algunas de las caracter�sticas que destacan son la siguientes:
\begin{milista}
	\item Todo el c�digo P2P se basa en la plataforma .NET.
	\item Los mensajes que se env�an entre los \textit{pares} son serializados como XML.
	\item Los \textit{pares} pueden compartir y acceder a distintos objetos.
	\item Tambi�n se ha implementado un servicio de descubrimiento utilizando .NET.
\end{milista}

\subsection{Peer-to-Peer Trusted Library}
\textit{Peer-to-Peer Trusted Library} (PtPTL) \cite{ptptl} es una biblioteca \textit{open-source} cuyo objetivo es profundizar en la seguridad de los sistemas P2P. Est� disponible tanto para sistemas basados en Windows como en sistemas basados en Linux. Sus principales caracter�sticas son las siguientes:
\begin{milista}
	\item Certificados digitales.
	\item Autenticaci�n de pares.
	\item Almacenamiento seguro.
	\item M�todos de codificaci�n sim�tricos y asim�tricos.
	\item Firmas digitales.
	\item Tarjetas digitales.
\end{milista}

Es importante apreciar que esta librer�a no es un \textit{toolkit} para desarrollar sistemas P2P, si no que es una biblioteca que permite dotar de "`confianza"' a dichos sistemas.
